A continuación desarrollamos la expresión para la probabilidad de obtener cara en la siguiente tirada de cualquiera de las tres monedas, aprovechando la información previa. Usamos fundamentalmente las tres propiedades vistas en clase: marginalización, reescritura de la probabilidad conjunta con probabilidad condicional, y la falacia del jugador.
  \begin{equation}
  \label{eq:4}
  \begin{aligned}
	P(cara_i|D) &= \int P(cara_i, \theta_i|D) d\theta_i \\
					&= \int P(cara_i|D, \theta_i)  P(\theta_i|D) d\theta_i \\
					&= \int P(cara_i|\theta_i)  P(\theta_i|D) d\theta_i \\ 
					&= \int \theta_i  (P(\theta_i, \alpha = i|D) + P(\theta_i, \alpha \neq i|D)) d\theta_i \\
					&= \int \theta_i  (P(\theta_i | \alpha = i,D) P(\alpha = i | D) + P(\theta_i | \alpha \neq i,D) P(\alpha \neq i | D)) d\theta_i \\
  \end{aligned}
  \end{equation}

En la última línea de esta ecuación podemos identificar las siguientes fuentes de incertidumbre:
\begin{itemize}
\item La incerteza inherente al \emph{rate} de la moneda;
\item Las incerteza de la \emph{posterior} de $\theta_i$ separando en los casos en que el modelo determina que $i$ es la moneda cargada y  los que no (recordar que el \emph{prior} utilizado para $\theta_i$ es condicional a este hecho);
\item La incerteza de determinar si una moneda está cargada o no (usando la \emph{posterior} de $\alpha$).
\end{itemize}

